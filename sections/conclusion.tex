\section{Conclusion}\label{sec:conclusion}
The app not only includes the weaknesses mentioned in the analysis, but also many positive aspects.
An example of a positive aspect is that the app tries to counteract some mentioned accessibility issues.
Other positive examples of the application are that Woebot is easily scalable and can be used immediately.
This counteracts the problems that there is a shortage of therapists in many places and appointments are made late.
However, by testing the various functionalities of the application, it becomes clear that Woebot can be improved in many areas.
The user group consists of people who have problems in their lives and are therefore looking for mental support.
This is the reason why the raised concerns should be taken into account and special care is required.
The main weakness of the application is the use of NLP.
Although this enables Woebot to form a higher therapeutic bond, it currently still leads to too many misunderstandings.
Furthermore, the illusion is created that Woebot truly understands the user.
It might be that a conversational agent could be a possible solution for offering therapeutic services in the future.
However, many social, ethical and technical problems must first be solved.
After these problems have been fixed, it must be evaluated whether the use of a conversational agent is the step in the right direction or alternatives are needed.
Therefore, further research in this area must be conducted.

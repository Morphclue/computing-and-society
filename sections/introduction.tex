\section{Introduction}
The need to create systems that support mental health is growing constantly. The World Health Organization estimated that around 322 million cases with depressive and 264 million cases with anxiety disorders existed in 2015\cite{who2017}. These two types of mental disorders are the most common and widely spread disorders. Studies suggest that these numbers have been continuously growing\cite{}\textcolor{red}{TODO}. Furthermore, it can be said that the current global COVID-19-pandemic contributes to increasing numbers in people suffering from mental disorders\cite{corona-mental}. Moreover, the demand for mental health professionals is way higher than the supply of these professionals\cite{indian-shortage, rural-shortage}. Often times it makes matters even worse that appointments are not immediate or affordable.\\

There are different methodologies in facing the shortage in mental health professionals and for dealing with depression or anxiety. A few examples are meditation, online classes or online forums. One approach is using a conversational agent to deal with the previously mentioned problems. \textcolor{red}{TODO - intro question?} \textcolor{red}{Intro to Woebot}\\

The following section introduces Woebot and how it is described by its creators. It also briefly explains key functionalities of the application. After that an analysis will follow, based on previous research and the functionalities of Woebot, in \autoref{sec:analysis}. This analysis will be subdivided in three categories: user, design and context. After that a conclusion will be drawn in \autoref{sec:conclusion}.

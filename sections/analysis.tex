\section{Analysis}\label{sec:analysis}
The analysis is divided into three different categories.
However, it should be said at this point that these categories will have overlaps.
Therefore, to avoid mentioning topics several times they will be mentioned only once.

\subsection{User}
The primary users for this application are persons with mental distress.
The question in this case is how is mental distress defined, or who is affected by it?
This question gets answered when looking into the core beliefs in the development of Woebot \cite{woebot-beliefs}.

\begin{quote}
    Everybody struggles sometimes. Cognitive distortions are something that everyone experiences in the context of strong emotion; it's part of being human.
\end{quote}

This quote shows clearly that not only persons with strong mental distress should consider using Woebot, but the application is for everyone.

% Definiton of mental distress?
% Design for everyone? (implicit) - Corinna Bath

% Single user
% Broad target

% CEO & Founder explanation

% Does it collect data, what does the app do with the data?
% How does it suggest to approach mental health? - Selfmanagement? Through Interaction?
% Norms and values: Culture sees mental problems as weakness -> Woebot doesn't solve this norm, it rather fights the weakness

% Human - AI Relations (Personification of Woebot?) - Concerns about therapeutic bond?

% GIFs + Emojis - Personal but might be annoying/ not for every target group

\subsection{Design}
% Mood-Tracker:
% Not possible to click on emojis
% What looks like a prediction is actually none

% Ai biased
% NLP - misunderstandings?
% Wrong calculations

\subsection{Context}
% Biometric sensor - privacy - phonesharing context
% Internet connection - Different contexts?
% Preload videos?

\section*{Abstract}
The application Woebot tries to improve mental health problems by using different therapeutic approaches combined with natural language processing and artificial intelligence.
This study highlights the different problems that can happen within this system by analyzing the application from a sociotechnical perspective.
At the same time the limits of the application are tested and assumptions made by the Woebot development team are refuted.
The result of the study is that Woebot has various social, ethical and technical problems.
In order to be able to provide therapeutic support, these problems must first be solved.
